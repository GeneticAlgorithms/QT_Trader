\documentclass[12pt]{article}
\usepackage[utf8]{inputenc}
\usepackage{amsmath, amssymb, amsthm}
\usepackage{graphicx}
\usepackage{hyperref}
\usepackage{natbib}
\usepackage{geometry}
\geometry{margin=1in}

\title{Adaptive Wave Model for Option Pricing:\\
A Comparison with the Black-Scholes Model}
\author{Student 1 Name \and Student 2 Name\\
\small University Name\\
\small Department Name}
\date{\today}

\begin{document}

\maketitle

\begin{abstract}
This paper implements and compares the adaptive nonlinear Schrödinger (NLS) wave model for option pricing with the standard Black-Scholes model. We focus on the shock-wave and soliton solutions of the NLS equation, which provide a quantum-mechanical perspective on option pricing. Through numerical implementation and comparison, we demonstrate the relationship between the wave function approach and traditional option pricing methods. Our results show [KEY FINDINGS]. The model offers an alternative framework for understanding option pricing dynamics, though proper parameter calibration remains a challenge.

\textbf{Keywords:} Option pricing, Nonlinear Schrödinger equation, Black-Scholes model, Wave functions, Financial modeling
\end{abstract}

\section{Introduction}

The Black-Scholes model \cite{black1973pricing} revolutionized option pricing by providing a closed-form solution for European options. However, the model has well-known limitations, including the assumption of constant volatility and the inability to capture market phenomena such as volatility smiles and skews. 

In this paper, we explore an alternative approach based on the adaptive nonlinear Schrödinger (NLS) equation, as proposed by Ivancevic \cite{ivancevic2009adaptive}. This wave-based model treats option prices as probability amplitudes, where the absolute square of the wave function represents the probability density function.

\subsection{Motivation}

The motivation for this work stems from:
\begin{itemize}
    \item Limitations of the Black-Scholes model in capturing market dynamics
    \item The potential of wave-based approaches to model financial markets
    \item The connection between quantum mechanics and probability in finance
\end{itemize}

\subsection{Paper Structure}

This paper is organized as follows: Section 2 reviews the Black-Scholes model. Section 3 introduces the NLS wave model. Section 4 describes our implementation. Section 5 presents results and comparison. Section 6 discusses findings and limitations. Section 7 concludes.

\section{Theoretical Background}

\subsection{Black-Scholes Model}

The Black-Scholes partial differential equation for option pricing is:
\begin{equation}
\frac{\partial u}{\partial t} = -\frac{1}{2}(\sigma s)^2 \frac{\partial^2 u}{\partial s^2} - rs \frac{\partial u}{\partial s} + ru
\end{equation}
where $u = u(t,s)$ is the option price, $\sigma$ is volatility, $r$ is the risk-free rate, and $s$ is the stock price.

For European call and put options, the closed-form solutions are:
\begin{align}
u_{\text{Call}}(s,t) &= s \mathcal{N}(d_1) e^{-T\delta} - k \mathcal{N}(d_2) e^{-rT} \\
u_{\text{Put}}(s,t) &= k \mathcal{N}(-d_2) e^{-rT} - s \mathcal{N}(-d_1) e^{-T\delta}
\end{align}
where $\mathcal{N}$ is the cumulative normal distribution, and $d_1$, $d_2$ are defined as in the standard Black-Scholes formula.

\subsection{Nonlinear Schrödinger Equation Model}

The adaptive NLS equation for option pricing is:
\begin{equation}
i\frac{\partial \psi}{\partial t} = -\frac{1}{2}\sigma \frac{\partial^2 \psi}{\partial s^2} - \beta |\psi|^2 \psi
\end{equation}
where $\psi(s,t)$ is the complex-valued wave function, and $|\psi(s,t)|^2$ represents the probability density function for the option price.

\subsubsection{Shock-Wave Solution}

The shock-wave (dark soliton) solution is:
\begin{equation}
\psi_2(s,t) = \pm \sqrt{\frac{-\sigma}{\beta}} \tanh(s - \sigma k t) \exp\left(i\left[ks - \frac{1}{2}\sigma t(2+k^2)\right]\right)
\end{equation}

\subsubsection{Soliton Solution}

The bright soliton solution is:
\begin{equation}
\psi_4(s,t) = \pm \sqrt{\frac{\sigma}{\beta}} \text{sech}(s - \sigma k t) \exp\left(i\left[ks - \frac{1}{2}\sigma t(k^2-1)\right]\right)
\end{equation}

\section{Implementation}

\subsection{Numerical Methods}

We implemented both models in Python using NumPy and SciPy. The Black-Scholes formulas were implemented directly from equations (2) and (3). The NLS solutions were computed using the analytical forms given in equations (5) and (6).

\subsection{Parameter Calibration}

For simplicity, we set $\beta = r$ (the interest rate) in most of our analysis. Full calibration would require optimization algorithms such as Levenberg-Marquardt, which is beyond the scope of this work.

\subsection{Comparison Methodology}

To compare the models, we:
\begin{enumerate}
    \item Calculate option prices using both models
    \item Compute error metrics (MSE, MAE)
    \item Visualize differences
    \item Analyze Greeks
\end{enumerate}

\section{Results}

\subsection{Option Price Comparison}

[Add your results here with figures]

Figure 1 shows the comparison between Black-Scholes and NLS model predictions for European call options. The NLS shock-wave solution shows qualitative agreement, though proper scaling is needed for quantitative comparison.

\subsection{Greeks Analysis}

[Add Greeks comparison]

\subsection{Parameter Sensitivity}

[Add parameter sensitivity results]

\section{Discussion}

\subsection{Key Findings}

\begin{itemize}
    \item The NLS model provides a novel perspective on option pricing
    \item The shock-wave solution shows promise for approximating Black-Scholes
    \item Parameter calibration is crucial for quantitative agreement
\end{itemize}

\subsection{Limitations}

\begin{itemize}
    \item Simplified parameter calibration ($\beta = r$)
    \item Need for proper normalization/scaling
    \item Computational complexity for full calibration
\end{itemize}

\subsection{Future Work}

\begin{itemize}
    \item Implement full parameter optimization
    \item Explore combined solutions
    \item Extend to stochastic volatility
    \item Validate with real market data
\end{itemize}

\section{Conclusion}

This paper has implemented and compared the adaptive NLS wave model with the Black-Scholes model. While the wave-based approach offers an interesting alternative framework, further work is needed on parameter calibration and validation with real market data. The model demonstrates the potential of applying quantum-mechanical concepts to financial modeling.

\section*{Acknowledgments}

We thank [advisor/instructor name] for guidance and the authors of the original paper for their theoretical contributions.

\bibliographystyle{plain}
\begin{thebibliography}{9}

\bibitem{black1973pricing}
Black, F., \& Scholes, M. (1973). The Pricing of Options and Corporate Liabilities. \textit{Journal of Political Economy}, 81(3), 637-659.

\bibitem{ivancevic2009adaptive}
Ivancevic, V. G. (2009). Adaptive–Wave Alternative for the Black–Scholes Option Pricing Model. arXiv:0911.1834v1 [q-fin.PR].

\bibitem{merton1973theory}
Merton, R. C. (1973). Theory of Rational Option Pricing. \textit{Bell Journal of Economics and Management Science}, 4, 141-183.

% Add more references as needed

\end{thebibliography}

\end{document}

