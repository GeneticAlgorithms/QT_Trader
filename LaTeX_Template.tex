\documentclass[12pt]{article}
\usepackage[utf8]{inputenc}
\usepackage{amsmath, amssymb, amsthm}
\usepackage{graphicx}
\usepackage{hyperref}
\usepackage{natbib}
\usepackage{geometry}
\geometry{margin=1in}

\title{Adaptive Wave Model for Option Pricing:\\
A Comparison with the Black-Scholes Model}
\author{Alexander Le \and Julian Yang\\
\small University of California, Berkeley\\
\small Department of Mathematics}
\date{\today}

\begin{document}

\maketitle

\begin{abstract}
This paper implements and compares the adaptive nonlinear Schrödinger (NLS) wave model for option pricing with the standard Black-Scholes model. We focus on the shock-wave and soliton solutions of the NLS equation, which provide a quantum-mechanical perspective on option pricing. Through numerical implementation and comparison, we demonstrate the relationship between the wave function approach and traditional option pricing methods. Our results show [KEY FINDINGS]. The model offers an alternative framework for understanding option pricing dynamics, though proper parameter calibration remains a challenge.

\textbf{Keywords:} Option pricing, Nonlinear Schrödinger equation, Black-Scholes model, Wave functions, Financial modeling
\end{abstract}

\section{Introduction}

The Black-Scholes model \cite{black1973pricing} revolutionized option pricing by providing a closed-form solution for European options. However, the model has well-known limitations, including the assumption of constant volatility and the inability to capture market phenomena such as volatility smiles and skews. 

In this paper, we explore an alternative approach based on the adaptive nonlinear Schrödinger (NLS) equation, as proposed by Ivancevic \cite{ivancevic2009adaptive}. This wave-based model treats option prices as probability amplitudes, where the absolute square of the wave function represents the probability density function.

\subsection{Motivation}

The motivation for this work stems from:
\begin{itemize}
    \item Limitations of the Black-Scholes model in capturing market dynamics
    \item The potential of wave-based approaches to model financial markets
    \item The connection between quantum mechanics and probability in finance
\end{itemize}

\subsection{Paper Structure}

This paper is organized as follows: Section 2 reviews the Black-Scholes model. Section 3 introduces the NLS wave model. Section 4 describes our implementation. Section 5 presents results and comparison. Section 6 discusses findings and limitations. Section 7 concludes.

\section{Theoretical Background}

\subsection{Black-Scholes Model}

The Black-Scholes model, developed by Black and Scholes \cite{black1973pricing} and Merton \cite{merton1973theory}, provides a framework for pricing European options. The model assumes that stock prices follow geometric Brownian motion and that markets are efficient.

The Black-Scholes partial differential equation for option pricing is:
\begin{equation}
\frac{\partial u}{\partial t} = -\frac{1}{2}(\sigma s)^2 \frac{\partial^2 u}{\partial s^2} - rs \frac{\partial u}{\partial s} + ru
\end{equation}
where $u = u(t,s)$ is the option price, $\sigma$ is volatility, $r$ is the risk-free rate, and $s$ is the stock price.

For European call and put options, the closed-form solutions are:
\begin{align}
u_{\text{Call}}(s,t) &= s \mathcal{N}(d_1) e^{-T\delta} - k \mathcal{N}(d_2) e^{-rT} \\
u_{\text{Put}}(s,t) &= k \mathcal{N}(-d_2) e^{-rT} - s \mathcal{N}(-d_1) e^{-T\delta}
\end{align}
where $\mathcal{N}$ is the cumulative normal distribution, and $d_1$, $d_2$ are defined as:
\begin{align}
d_1 &= \frac{\ln(s/k) + T(r - \delta + \sigma^2/2)}{\sigma\sqrt{T}} \\
d_2 &= d_1 - \sigma\sqrt{T}
\end{align}

\subsection{Limitations of Black-Scholes}

While the Black-Scholes model revolutionized option pricing, it has several well-documented limitations:
\begin{itemize}
    \item Constant volatility assumption, which contradicts observed volatility smiles and skews
    \item Inability to capture market crashes and extreme events
    \item Assumption of continuous trading and no transaction costs
    \item Normal distribution assumption for returns, which doesn't capture fat tails
\end{itemize}

These limitations motivate the exploration of alternative models, such as the wave-based approach.

\subsection{Nonlinear Schrödinger Equation Model}

The adaptive NLS equation for option pricing, as proposed by Ivancevic \cite{ivancevic2009adaptive}, treats option prices using quantum-mechanical wave functions. The model is formally defined as:
\begin{equation}
i\frac{\partial \psi}{\partial t} = -\frac{1}{2}\sigma \frac{\partial^2 \psi}{\partial s^2} - \beta |\psi|^2 \psi
\end{equation}
where $\psi(s,t)$ is the complex-valued wave function, and $|\psi(s,t)|^2$ represents the probability density function for the option price. Here, $\sigma$ is the volatility (dispersion coefficient) and $\beta = \beta(r,w)$ is the adaptive market potential.

\subsubsection{Shock-Wave Solution}

The shock-wave (dark soliton) solution, which shows the best agreement with Black-Scholes, is:
\begin{equation}
\psi_2(s,t) = \pm \sqrt{\frac{-\sigma}{\beta}} \tanh(s - \sigma k t) \exp\left(i\left[ks - \frac{1}{2}\sigma t(2+k^2)\right]\right)
\end{equation}
where $k$ is the wave number. The probability density is given by:
\begin{equation}
|\psi_2(s,t)|^2 = \frac{\sigma}{|\beta|} \tanh^2(s - \sigma k t)
\end{equation}

\subsubsection{Soliton Solution}

The bright soliton solution is:
\begin{equation}
\psi_4(s,t) = \pm \sqrt{\frac{\sigma}{\beta}} \text{sech}(s - \sigma k t) \exp\left(i\left[ks - \frac{1}{2}\sigma t(k^2-1)\right]\right)
\end{equation}
with probability density:
\begin{equation}
|\psi_4(s,t)|^2 = \frac{\sigma}{\beta} \text{sech}^2(s - \sigma k t)
\end{equation}

\subsection{Option Greeks}

The Greeks measure the sensitivity of option prices to various parameters. For the NLS model, we can derive Greeks from the wave function solutions. The key Greeks are:
\begin{itemize}
    \item \textbf{Delta} ($\Delta$): Sensitivity to stock price changes, $\Delta = \partial u/\partial s$
    \item \textbf{Gamma} ($\Gamma$): Rate of change of Delta, $\Gamma = \partial^2 u/\partial s^2$
    \item \textbf{Vega} ($\nu$): Sensitivity to volatility, $\nu = \partial u/\partial \sigma$
    \item \textbf{Theta} ($\Theta$): Time decay, $\Theta = \partial u/\partial t$
    \item \textbf{Rho} ($\rho$): Sensitivity to interest rate, $\rho = \partial u/\partial r$
\end{itemize}

\section{Examination of Data and Model Setup}

In this section, we describe our data sources, parameter selection, and the methodology for comparing the NLS wave model with the Black-Scholes model.

\subsection{Data and Parameters}

For our analysis, we use synthetic and standard market parameters:
\begin{itemize}
    \item Stock price range: $S \in [50, 150]$ (normalized around strike price $K = 100$)
    \item Time to maturity: $T = 1.0$ year
    \item Risk-free rate: $r = 0.05$ (5\% annually)
    \item Volatility: $\sigma = 0.3$ (30\% annually)
    \item Wave number: $k = 1.2$ (selected to match typical market behavior)
    \item Market potential: $\beta = r = 0.05$ (simplified calibration)
\end{itemize}

\subsection{Parameter Calibration}

For this study, we use a simplified calibration approach where $\beta = r$, the interest rate. This simplification allows us to focus on the core comparison between models. Full calibration would require optimization algorithms such as Levenberg-Marquardt to fit the adaptive potential $\beta(r,w)$ with error functions, as described in the original paper \cite{ivancevic2009adaptive}.

\section{Implementation}

\subsection{Numerical Methods}

We implemented both models in Python using NumPy and SciPy libraries. The Black-Scholes formulas were implemented directly from equations (2) and (3), using the error function from SciPy for the cumulative normal distribution.

The NLS solutions were computed using the analytical forms given in equations (5) and (6). For the shock-wave solution, we compute:
\begin{equation}
|\psi_2(s,t)|^2 = \frac{\sigma}{|\beta|} \tanh^2(s - \sigma k t)
\end{equation}
and for the soliton solution:
\begin{equation}
|\psi_4(s,t)|^2 = \frac{\sigma}{\beta} \text{sech}^2(s - \sigma k t)
\end{equation}

\subsection{Comparison Methodology}

To compare the models, we:
\begin{enumerate}
    \item Calculate option prices using both Black-Scholes and NLS models
    \item Normalize NLS probability densities to option price scales for comparison
    \item Compute error metrics: Mean Squared Error (MSE) and Mean Absolute Error (MAE)
    \item Visualize differences across different stock price ranges
    \item Analyze and compare Greeks from both models
    \item Perform parameter sensitivity analysis
\end{enumerate}

\subsection{Scaling and Normalization}

A key challenge in comparing the NLS model with Black-Scholes is that the NLS model produces probability densities $|\psi|^2$, while Black-Scholes produces option prices. For comparison purposes, we scale the NLS PDFs to match the scale of Black-Scholes prices. This is a simplified approach; full calibration would require optimization to match specific option prices.

\section{Results}

\subsection{Option Price Comparison}

We compare the Black-Scholes call option prices with the NLS shock-wave solution across a range of stock prices from \$70 to \$130, with strike price $K = 100$.

The NLS shock-wave solution shows qualitative agreement with Black-Scholes in terms of shape and general behavior. Both models predict higher option prices when the stock price is above the strike price (in-the-money) and lower prices when below (out-of-the-money).

However, quantitative comparison reveals differences that highlight the need for proper parameter calibration. The error analysis shows that while the models follow similar trends, the absolute values differ, particularly in the at-the-money region where $S \approx K$.

\subsubsection{Error Metrics}

We compute the following error metrics between the scaled NLS model and Black-Scholes:
\begin{itemize}
    \item Mean Squared Error (MSE): Measures overall deviation
    \item Mean Absolute Error (MAE): Provides average absolute difference
    \item Maximum Error: Identifies regions of largest discrepancy
\end{itemize}

These metrics help quantify the agreement between models and identify areas where calibration could improve the fit.

\subsection{Greeks Analysis}

We calculate and compare the option Greeks from both models. The Greeks provide important risk management information and reveal how each model responds to parameter changes.

\subsubsection{Delta Comparison}

Delta measures the sensitivity of option price to stock price changes. Both models show similar Delta behavior: increasing as the stock price rises for call options, with values between 0 and 1.

\subsubsection{Gamma Comparison}

Gamma measures the rate of change of Delta. The NLS model shows a more concentrated Gamma profile compared to Black-Scholes, which may reflect the localized nature of the wave function solutions.

\subsubsection{Vega Comparison}

Vega measures sensitivity to volatility changes. Both models show that option prices increase with volatility, though the magnitude differs between models.

\subsection{Parameter Sensitivity Analysis}

We analyze how different parameters affect the NLS model predictions.

\subsubsection{Volatility Sensitivity}

We test the NLS model with volatility values $\sigma \in \{0.1, 0.2, 0.3, 0.4, 0.5\}$. Higher volatility leads to:
\begin{itemize}
    \item Wider probability distributions
    \item More spread in the wave function
    \item Greater sensitivity to price changes
\end{itemize}

This behavior is consistent with financial intuition: higher volatility implies greater uncertainty and wider price distributions.

\subsubsection{Market Potential Sensitivity}

We test the effect of market potential $\beta \in \{0.01, 0.03, 0.05, 0.07, 0.10\}$. The market potential parameter controls the strength of the nonlinear term in the NLS equation. Higher values of $\beta$ lead to:
\begin{itemize}
    \item More localized wave functions
    \item Sharper probability distributions
    \item Different scaling of the solutions
\end{itemize}

\subsection{Performance Metrics}

To evaluate the models, we compute several performance metrics:
\begin{itemize}
    \item \textbf{Computational Efficiency}: The NLS model uses analytical solutions, making it computationally efficient
    \item \textbf{Model Flexibility}: The wave function approach offers different solution types (shock-wave, soliton)
    \item \textbf{Calibration Complexity}: Full calibration requires optimization, which is more complex than Black-Scholes
\end{itemize}

\section{Discussion}

\subsection{Key Findings}

Our analysis reveals several important findings:

\begin{enumerate}
    \item \textbf{Qualitative Agreement}: The NLS shock-wave solution shows strong qualitative agreement with Black-Scholes in terms of shape and general behavior. Both models predict similar option price curves.
    
    \item \textbf{Quantitative Differences}: While the shapes are similar, quantitative differences exist, particularly in the at-the-money region. This highlights the importance of proper parameter calibration.
    
    \item \textbf{Model Flexibility}: The NLS model offers multiple solution types (shock-wave, soliton, Jacobi functions), providing flexibility that Black-Scholes lacks.
    
    \item \textbf{Greeks Behavior}: The Greeks from the NLS model show similar trends to Black-Scholes but with some differences in magnitude and distribution, particularly for Gamma.
    
    \item \textbf{Parameter Sensitivity}: The NLS model responds intuitively to parameter changes, with volatility and market potential affecting the solutions in expected ways.
\end{enumerate}

\subsection{Advantages of the NLS Model}

\begin{itemize}
    \item \textbf{Novel Perspective}: Provides a quantum-mechanical wave function approach to option pricing
    \item \textbf{Multiple Solutions}: Offers different solution types that can be combined or selected based on market conditions
    \item \textbf{Adaptive Potential}: The market potential $\beta(r,w)$ can be made adaptive, allowing the model to learn from market data
    \item \textbf{Computational Efficiency}: Analytical solutions enable fast computation without numerical PDE solvers
\end{itemize}

\subsection{Limitations and Challenges}

\begin{itemize}
    \item \textbf{Parameter Calibration}: Full calibration requires optimization algorithms (Levenberg-Marquardt) and is more complex than Black-Scholes parameter estimation
    
    \item \textbf{Scaling and Normalization}: The NLS model produces probability densities that need proper scaling to match option prices
    
    \item \textbf{Market Potential Form}: The simplified $\beta = r$ approach may not capture all market dynamics; the full adaptive potential with error functions is more complex
    
    \item \textbf{Validation}: This study uses synthetic/standard parameters; validation with real market data is needed
    
    \item \textbf{Interpretation}: The wave function interpretation, while mathematically sound, requires understanding of quantum mechanics concepts
\end{itemize}

\subsection{Comparison with Black-Scholes}

The NLS model offers an alternative framework to Black-Scholes with both similarities and differences:

\textbf{Similarities:}
\begin{itemize}
    \item Both models produce option price curves with similar shapes
    \item Both respond similarly to changes in volatility and stock price
    \item Greeks show similar trends and behaviors
\end{itemize}

\textbf{Differences:}
\begin{itemize}
    \item NLS uses wave functions and probability amplitudes; Black-Scholes uses direct pricing formulas
    \item NLS offers multiple solution types; Black-Scholes has a single formula
    \item NLS can be made adaptive; Black-Scholes uses fixed parameters
    \item NLS requires calibration; Black-Scholes parameters are more straightforward to estimate
\end{itemize}

\subsection{Future Work}

Several directions for future research emerge from this study:

\begin{enumerate}
    \item \textbf{Full Parameter Calibration}: Implement the Levenberg-Marquardt algorithm to optimize the adaptive market potential $\beta(r,w)$ with error functions, as described in the original paper.
    
    \item \textbf{Combined Solutions}: Explore the combined shock-wave and soliton solution (equation 18) to smooth out kinks and improve fit quality.
    
    \item \textbf{Stochastic Volatility}: Extend the model to include stochastic volatility using the Manakov system of coupled NLS equations.
    
    \item \textbf{Real Market Data Validation}: Test the model on actual option market data to validate its predictive power.
    
    \item \textbf{Hebbian Learning}: Implement the Hebbian learning dynamics for adaptive parameter estimation.
    
    \item \textbf{Multi-Asset Extension}: Explore whether the wave function approach can be extended to portfolio optimization and multi-asset scenarios.
    
    \item \textbf{Performance Optimization}: Develop more efficient numerical methods for cases where analytical solutions are not available.
\end{enumerate}

\section{Conclusion}

This paper has implemented and compared the adaptive NLS wave model with the standard Black-Scholes model for option pricing. Our analysis demonstrates that the NLS shock-wave solution shows qualitative agreement with Black-Scholes predictions, with both models producing similar option price curves and Greeks behavior.

The wave function approach offers a novel perspective on option pricing, treating prices as probability amplitudes rather than direct market values. This quantum-mechanical framework provides flexibility through multiple solution types and the potential for adaptive learning.

However, our results also highlight the importance of proper parameter calibration. While we used a simplified approach ($\beta = r$), full calibration with optimization algorithms would be necessary for quantitative agreement with market prices. The scaling and normalization of probability densities to option prices also requires careful consideration.

The NLS model shows promise as an alternative to Black-Scholes, particularly in its ability to offer multiple solution types and adaptive potential. However, further work is needed on calibration, validation with real market data, and extension to more complex scenarios such as stochastic volatility.

This study contributes to the growing field of applying physics-based methods to financial modeling, demonstrating that wave mechanics can provide insights into option pricing dynamics. The model's flexibility and computational efficiency make it an interesting direction for future research in quantitative finance.

\section*{Acknowledgments}

We thank [advisor/instructor name] for guidance and the authors of the original paper for their theoretical contributions.

\bibliographystyle{plain}
\begin{thebibliography}{9}

\bibitem{black1973pricing}
Black, F., \& Scholes, M. (1973). The Pricing of Options and Corporate Liabilities. \textit{Journal of Political Economy}, 81(3), 637-659.

\bibitem{ivancevic2009adaptive}
Ivancevic, V. G. (2009). Adaptive–Wave Alternative for the Black–Scholes Option Pricing Model. arXiv:0911.1834v1 [q-fin.PR].

\bibitem{merton1973theory}
Merton, R. C. (1973). Theory of Rational Option Pricing. \textit{Bell Journal of Economics and Management Science}, 4, 141-183.

% Add more references as needed

\end{thebibliography}

\end{document}

